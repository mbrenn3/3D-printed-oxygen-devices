% Template for PLoS
% Version 2.0 July 2014
%
% To compile to pdf, run:
% latex plos.template
% bibtex plos.template
% latex plos.template
% latex plos.template
% dvipdf plos.template
%
% % % % % % % % % % % % % % % % % % % % % %
%
% -- IMPORTANT NOTE
%
% Be advised that this is merely a template 
% designed to facilitate accurate translation of manuscript content 
% into our production files. 
%
% This template contains extensive comments intended 
% to minimize problems and delays during our production 
% process. Please follow the template 
% whenever possible.
%
% % % % % % % % % % % % % % % % % % % % % % % 
%
% Once your paper is accepted for publication and enters production, 
% PLEASE REMOVE ALL TRACKED CHANGES in this file and leave only
% the final text of your manuscript.
%
% DO NOT ADD EXTRA PACKAGES TO THIS TEMPLATE unless absolutely necessary.
% Packages included in this template are intentionally
% limited and basic in order to reduce the possibility
% of issues during our production process.
%
% % % % % % % % % % % % % % % % % % % % % % %
%
% -- FIGURES AND TABLES
%
% DO NOT INCLUDE GRAPHICS IN YOUR MANUSCRIPT
% - Figures should be uploaded separately from your manuscript file. 
% - Figures generated using LaTeX should be extracted and removed from the PDF before submission. 
% - Figures containing multiple panels/subfigures must be combined into one image file before submission.
% See http://www.plosone.org/static/figureGuidelines for PLOS figure guidelines.
%
% Tables should be cell-based and may not contain:
% - tabs/spacing/line breaks within cells to alter layout
% - vertically-merged cells (no tabular environments within tabular environments, do not use \multirow)
% - colors, shading, or graphic objects
% See http://www.plosone.org/static/figureGuidelines#tables for table guidelines.
%
% For sideways tables, use the {rotating} package and use \begin{sidewaystable} instead of \begin{table} in the appropriate section. PLOS guidelines do not accomodate sideways figures.
%
% % % % % % % % % % % % % % % % % % % % % % % %
%
% -- EQUATIONS, MATH SYMBOLS, SUBSCRIPTS, AND SUPERSCRIPTS
%
% IMPORTANT
% Below are a few tips to help format your equations and other special characters according to our specifications. For more tips to help reduce the possibility of formatting errors during conversion, please see our LaTeX guidelines at http://www.plosone.org/static/latexGuidelines
%
% Please be sure to include all portions of an equation in the math environment, and for any superscripts or subscripts also include the base number/text. For example, use $mathrm{mm}^2$ instead of mm$^2$ (do not use \textsuperscript command).
%
% DO NOT USE the \rm command to render mathmode characters in roman font, instead use $\mathrm{}$
% For bolding characters in mathmode, please use $\mathbf{}$ 
%
% Please add line breaks to long equations when possible in order to fit our 2-column layout. 
%
% For inline equations, please do not include punctuation within the math environment unless this is part of the equation.
%
% For spaces within the math environment please use the \; or \: commands, even within \text{} (do not use smaller spacing as this does not convert well).
%
%
% % % % % % % % % % % % % % % % % % % % % % % %



\documentclass[10pt]{article}



% amsmath package, useful for mathematical formulas
\usepackage{amsmath}
% amssymb package, useful for mathematical symbols
\usepackage{amssymb}

% cite package, to clean up citations in the main text. Do not remove.
\usepackage{cite}

\usepackage{hyperref}

% line numbers
\usepackage{lineno}

% ligatures disabled
\usepackage{microtype}
\DisableLigatures[f]{encoding = *, family = * }

% rotating package for sideways tables
%\usepackage{rotating}

% If you wish to include algorithms, please use one of the packages below. Also, please see the algorithm section of our LaTeX guidelines (http://www.plosone.org/static/latexGuidelines) for important information about required formatting.
%\usepackage{algorithmic}
%\usepackage{algorithmicx}

% Use doublespacing - comment out for single spacing
%\usepackage{setspace} 
%\doublespacing

\usepackage[version=3]{mhchem}

% Text layout
\topmargin 0.0cm
\oddsidemargin 0.5cm
\evensidemargin 0.5cm
\textwidth 16cm 
\textheight 21cm

% Bold the 'Figure #' in the caption and separate it with a period
% Captions will be left justified
\usepackage[labelfont=bf,labelsep=period,justification=raggedright]{caption}

% Use the PLoS provided BiBTeX style
\bibliographystyle{plos2009}

% Remove brackets from numbering in List of References
\makeatletter
\renewcommand{\@biblabel}[1]{\quad#1.}
\makeatother


% Leave date blank
\date{}

\pagestyle{myheadings}

%% Include all macros below. Please limit the use of macros.

%% END MACROS SECTION


\begin{document}


% Title must be 150 characters or less
\begin{flushleft}
{\Large
\textbf{3D-Printed Microfluidic Oxygen Control Devices}
}
% Insert Author names, affiliations and corresponding author email.
\\
Martin D. Brennan$^{1}$, 
Megan L. Rexius-Hall$^{1}$, 
David T. Eddington$^{1,\ast}$
\\
\bf{1} Dept of Bioengineering, University of Illinois at Chicago, Chicago, Illinois, USA
\\
%\bf{2} Author2 Dept/Program/Center, Institution Name, City, State, Country
%\\
%\bf{3} Author3 Dept/Program/Center, Institution Name, City, State, Country
%\\
$\ast$ E-mail: Corresponding dte@uic.edu
\end{flushleft}

% Please keep the abstract between 250 and 300 words
\section*{Abstract}

Recently 3D printing has emerged as a method for directly printing complete microfluidic chips, 
although printing materials have been limited to non-oxygen permeable materials.
We demonstrate the addition of gas permeable PDMS (Polydimethylsiloxane) membranes to 3D printed microfluidic devices as a means to enable oxygen control cell culture studies.
This was demonstrated on an expanded 24 well version of our previous device which was for a 6-well plate.
The direct printing allows integrated distribution channels and device geometries not possible with traditional planar lithography.
With this device oxygen was able to be controlled to 4 different conditions across a 24 well plate demonstrating regulation of oxygen sensitive transcriptional protein in cancer cells.
This is the first 3D printed device that can be functionalized to control oxygen in cell culture.
%Two devices are presented and characterized.


% Please keep the Author Summary between 150 and 200 words
% Use first person. PLOS ONE authors please skip this step. 
% Author Summary not valid for PLOS ONE submissions.   
\section*{Author Summary}



\section*{Introduction}

Here we report on the development of 3D printed microfluidic devices for the control of oxygen in cell culture microenvironments.
We demonstrate a device that nests into a 24 well culture plate to control gas in each row of the plate independently of the incubator's condition.
This expands on our previous work of a similar device for 6-well plates\cite{Oppegard2009,Oppegard2010}.
The ability to independentaly control oxygen across each row of the plate enables more efficient experiments as a separate incubator or hypoxic chamber is not needed for each condition.

3D printing of microfluidic devices enables rapid, one-step fabrication of complex designs infeasible to make with planar lithography and replica molding techniques.
In addition, planar lithography is time consuming, requires specialized equipment and facilities, and has a high failure rate.
It is not unusual for microfluidic labs to make ten microfluidic devices to guarantee one will work properly.
On the other hand, 3D CAD printing allows for unambiguous specifications and nearly eliminates time and effort spent on fabrication which may be outsourced to a 3D printing company for around \$100/device \cite{Au2014}.
3D printing also allows integration of complex geometries not possible with planar lithography, such as hose barbs and luer fittings.
Dissemination and distributed production is also vastly simplified due to easy sharing of the design as a CAD file.
Due to these inherent advantages 3D printing has emerged as a method for directly printing complete microfluidic devices \cite{Au2014, Shallan 2014, Chen2014, Erkal2014,Bhargava2014}. 
Many prototypical microfluidic device features have been recreated with 3D printing as a proof of concept for this new fabrication technique \cite{Au2014, Shallan2014} including modular re-configurable units\cite{Bharagava2014}.
3D printed devices have been used for inexpensive and high-throughput reactionware,\cite{Kitson2014}, for a measuring dopamine and ATP levels in biological samples with an integrated electrode\cite{Erkal2014}, as well as a device that allows flow through measurements of ATP to be taken with a plate reader\cite{Chen2014}.

Printing is currently limited in choice of materials compatible with the process and include many proprietary formulations, regardless these have been successfully used in a variety of applications.
As of yet there is no widely available methods or materials to facilitate direct printing of gas permeable materials although this area is being developed\cite{Femmer2014}.
Microfluidic cell culture devices are most commonly cast in PDMS.
PDMS is a convenient material for cell studies due to it's biocompatibility, optical properties, and gas permeability, facilitating oxygen control of cell environments.

Oxygen control in cells studies is often overlooked by researchers, but important for mimicking conditions experienced by cells in vivo.
Typically cell culture studies are preformed at 21\% oxygen, atmospheric oxygen conditions, although levels that cells experience in vivo are usually less.
For example tumors, in general, are hypoxic as they rapidly out grow their vasculature creating a hypoxic inner region. 
To better study the role of oxygen levels in cancer gene expression a gas controlled culture system is required. ----I have to cite all this----

Previously, we developed multiwell inserts for 6-well plates that controlled oxygen in a standard off-the self plate \cite{Oppegard2009, Oppegard2010}.
These devices were completely cast from PDMS and fabricated in a multiple bonding procedure.
In addition, tubing was used to connect each pillar which became quite cumbersome when moving to a newer 24-well version.
Utilizing 3D printing for the fabrication allowed further features to be incorporated such as integrated distribution channels to eliminate the connection tubings between wells and hose barbs for better tubing connections, while eliminating fabrication time and failure rate.
Although available 3D printable materials are non-gas-permeable, the addition of a PDMS membrane following printing the passive microfluidic network is a simple addition allowing gas transfer. 
In this case, having a gas-impermeable material for the bulk is advantageous as it reduces unwanted gas transfer. 
The previous PDMS devices were parylene coated along the convection channels to eliminate exchange of gas from the PDMS bulk, which again added to their complexity and is now longer needed.
Convection and distribution of the gas is done in the impermeable material eliminating dilution of the gas before reaching the diffusion layer. 

% You may title this section "Methods" or "Models". 
% "Models" is not a valid title for PLoS ONE authors. However, PLoS ONE
% authors may use "Analysis" 
\section*{Materials and Methods}


\subsection*{Design of Insert}

The device was designed to integrate with a multiwell format, specifically an off-the-shelf 24 well plate. 
The 24 well plate insert is designed to control gas in 6 wells of a 24 well culture plate from one input, borrowing the working principle of previous work\cite{oppegard2010} and also incorporates an integrated distribution network and hose barbs to simplify device operation.
The pillars extend into each well leaving a $\sim$500 $\mu$m gap for media between the diffusion membrane and the culture surface at the bottom of the well.
Diffusion occurs rapidly across this gap allowing control of the dissolved gas environment around the cells. 
A distribution network stems from the central input that equalizes the flow along each path length by varying the channel width to the proximal, intermediate, and distal wells (Figure \ref{figure1}.

The device also features a pipe within a pipe design so that gas flow enters and leaves the diffusion area in a uniform, and symmetrical flow pattern, which would not be possible with standard lithography and demonstrates the capabilities of 3D printing (Figure \ref{figure1}.
Again the membranes are adhered with a thin layer of PDMS on the membrane and allowed to cure in place.

The device was printed by Fineline Prototyping in Watershed XC using stereolithography.
CAD models were designed and printed with microfluidic delivery channels, and then completed by adhering a gas permeable membrane of PDMS to enable diffusion of gas to the culture area using uncured PDMS as an adhesive. 

\subsection*{Oxygen characterization}

Oxygen was measured with PtOEPK (Pt(II) Octaethylporphine ketone) planar sensors that were fixed to the bottom of a 24 well plate. 
The intensity of the sensor was measured and correlated to the concentration of oxygen with Stern-Volmer analysis (Figure \ref{piller-well-render-figure}.
Gas was perfused through the device with negative pressure to reduce membrane debonding and was also found to reduce the number of bubbles formed in the culture chamber.
The inlet tubing was placed in a cone with with a flow that exceed the vacuum flow rate so that the vacuum was always pulling in the gas of interest and not the room air.
All gas tanks used in the oxygen characterization were 5\% CO2 in addition to the desired gas mix in anticipation of cell culture experiments as CO2 can alter the fluorescence of the PtOEPK.
Intensity measurments were taken every five minutes for each well.
Initially the inlets are fed with a 5\% \ce{CO2}, bal. air tank until the intensity stableized.
The inlets are then switched to the contol gas of either 0, 5 10 or 21\% \ce{O2}, each with 5\% \ce{CO2} and balanced with \ce{N2}.
Intensity measurements are then taken every five minutes for six hours.
The this entire procedure was reapeated three times for and N of three. 

\subsection*{Cell culture and hypoxia assay}

Cancer cells were seeded in a 24 well plate.
When they reached $\sim$70\% confluency the 24-well insert is placed in the plate and the gasses are perfused through the device in the same scheme as in the oxygen characterization.
Each row of six wells experiences either 0, 5, 10 or 21\% \ce{O2}, with 5\% \ce{CO2} and balanced nitrogen.
The assmebly is placed in an incubator at $37\,^{\circ}\mathrm{C}$ for six hours. 


% Results and Discussion can be combined.
\section*{Results and Discussion}

\subsection*{Oxygen Control}
Oxygen control in the 24-well insert device was quantified with a platinum based (PtOEPK) planar oxygen sensor placed at the bottom of the well, as shown in Figure \ref{figure1}.
Oxygen was able to be controlled in each of the 6 wells equally from one gas input demonstrating the distribution network worked effectively.
The device reaches steady state in 30 minutes and can then hold the oxygen level near 0\% \ce{O2} indefinitely, while 95\% \ce{N2}, 5\% \ce{CO2} is pumped through the device. 
A different gas condition can be used in each sepereate iteration of the 6-well unit allowing 6 tecnical replicants of up to 4 different gas conditions in one 24 well plate\ref{figure2}.

\subsection*{Bioverifcation}
The devices ablility to control the gas enviroment is demonstrated with hypoxia influenced expression of mRNA in cancer cells\ref{fig3}.
Oxygen levels of 21, 10, 5, and 0\% are applied to each set of 6 wells. 

More mRNA stuff for megan to expand \ref{figure3}

This new iteration of the multi-well insert for oxygen control maintains the convient features of the previous design.
Oxygen contol is preformed in an off-the-shelf culture plate allowing standard protocols to be used for analysis assays or imaging where additional protocols are often requried when cells are seeded within microfuildic chips.

This 3D-printed device improves on our previous design in a number of ways.
First, the fabrication is greatly simplified. 
No microfabrication, PDMS molding or Parylene coating is required.
3D-printing also enables higher levels of intricacy without any additional fabrication allowing us to include the distribution network and the 'pipe within a pipe' design. 
Moving forward with additive manufacturing of microfluidic devices is desirable. 
A comprable PDMS device would require several additional microfabriated layers to be manually aligned to reporduce just the distribution newtork.
Additionally, the non-premanat addheasion of membranes to the 3D-part allow them to be removed and replaced if they are damaged.
In PDMS devices the layers are bonded permantly via plasma bonding so damage to a membrane is not repairable.

3D-printing allows this design to be shared directly and modified by an end user without requireing any microfabrication techniques.
An on device mixer that could supply different levels of mixing between two sources of gas could reduce the number of tanks required.
A mixer could also allow more conditions in a single plate perhaps even 1 per well or even gradients within wells.
Even hight format plates such as 96 well plates could be used.
Gas connections could be reloated to allow stacking of device in incubators.

% We only support three levels of headings, please do not create a heading level below \subsubsection.
\subsection*{Subsection 1}

\subsubsection*{SubSubsection 1.1}

\subsection*{Subsection 2}

%\section*{Discussion}

\section*{Conclusion}
3D printing microfluidic chips have been limited to non oxygen permeable materials.
We demonstrate oxygen control in 3D printed devices the addition of a gas permeable PDMS membrane. 
Oxygen control is demonstrated in a 24 well plate and PCR analysis of X in Y cells shows the device effectively controls oxygen as expected.
3D printing allows complex designs, integrated tubing connectors, and is comparable in price to standard PDMS fabrication.
This technique represents a bridge to commercialization where robust devices can be more easily shared and disseminated.
While injection molding, hot embossing, or other industrial processes are cheaper when making hundreds to thousands of devices, it is not practical to make a injection mold when making tens to hundreds of devices.
In addition, PDMS fabrication would be too time consuming, expensive, and the failure rate would be unacceptable.
3D printing is a perfect solution to these device fabrication needs.


% Do NOT remove this, even if you are not including acknowledgments.

\section*{Acknowledgments}

This work funded by NSF 1253060.


%\section*{References}

\bibliography{references}

% Either type in your references using
% \begin{thebibliography}{}
% \bibitem{}
% Text
% \end{thebibliography}
%
% OR
%
% Compile your BiBTeX database using our plos2009.bst
% style file and paste the contents of your .bbl file
% here.
% 

\section*{Figure Legends}
% This section is for figure legends only, do not include
% graphics in your manuscript file.
%
%\begin{figure}
%\caption{
%{\bf Bold the first sentence.}  Rest of figure caption.  
%}
%\label{Figure_label}
%\end{figure}

% copy and paste for figures
%\begin{figure}
%\includegraphics[scale=0.2]{image.jpg} % remove for manuscript submission
%\caption{
%{\bf Bold the first sentence.}  Rest of figure caption.  
%}
%\label{Figure_label}
%\end{figure}

\begin{figure}
%\includegraphics[scale=0.2]{../presentation-figures/piller-well-render-with-arrows.png} % remove for manuscript submission
\caption{
{\bf CAD rendering of the piller bottom.} The 'Pipe with in a pipe' flow pattern delivers gas to the bottom of each well where it diffuses to the culture.  
}
\label{piller-well-render-figure}
\end{figure}

\begin{figure}
%\includegraphics[scale=0.05]{../presentation-figures/24well.JPG} % remove for manuscript submission
%\includegraphics[scale=0.05]{../presentation-figures/printed-network.JPG} % remove for manuscript submission
\caption{
{\bf Photographs of the printed device.}  The printed device with membranes adhered. The distribution network is printed completely in resin.
}
\label{device-photos-figure}
\end{figure}

\begin{figure}
%\includegraphics[scale=0.3]{../presentation-figures/6-well-plot.png} % remove for manuscript submission
%\includegraphics[scale=0.3]{../presentation-figures/24-well-plot.png} % remove for manuscript submission
\caption{
{\bf Oxygen Characterization.}  Time course data of oxygen being evacuated from the culture area as 0\% oxygen gas is perfused through the device. Each 6 well row of the plate can be controlled independently.  
}
\label{oxygen-char-figure}
\end{figure}

\begin{figure}
%\includegraphics[scale=0.2]{image.jpg} % remove for manuscript submission
\caption{
{\bf PCR Data.}  Not finished yet.  
}
\label{pcr-data}
\end{figure}



\section*{Tables}
% 
% See introductory notes if you wish to include sideways tables.
%
% NOTE: Please look over our table guidelines at http://www.plosone.org/static/figureGuidelines#tables to make sure that your tables meet our requirements. Certain types of spacing, cell merging, and other formatting tricks may have unintended results and will be returned for revision.
%
%\begin{table}[!ht]
%\caption{
%\bf{Table title}}
%\begin{tabular}{|c|c|c|}
%table information
%\end{tabular}
%\begin{flushleft}Table caption
%\end{flushleft}
%\label{tab:label}
% \end{table}

\section*{Supporting Information Legends}
%
% Please enter your Supporting Information captions below in the following format:
%\item{\bf Figure SX. Enter mandatory title here.} Enter optional descriptive information here.
% 
%\begin{description}
%\item {\bf}
%\item {\bf}
%\end{description}

\end{document}

